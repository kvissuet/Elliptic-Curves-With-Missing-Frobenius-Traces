\section{Notation and Group-theoretic preliminaries}

In this section, we gather notation and preliminary lemmas.  Here and henceforth in the paper, given an open subgroup $G \subseteq \GL_2(\hat{\mbz})$, we put
\begin{equation*}
\begin{split}
m_G &:= \level_{\GL_2}(G) \\
m_S &:= \level_{\SL_2}(G),
\end{split}
\end{equation*}
defined as in \eqref{defofGL2levelandSL2level}.
For any open subgroup $S \subseteq \SL_2(\hat{\mbz})$, we will also denote by $m_S$ its $\SL_2$-level.  Also, for any such subgroups we maintain the notation from the introduction:
\begin{equation} \label{defoofGtildeandStilde}
\begin{split}
\tilde{G} &:= \langle -I, G \rangle, \\
\tilde{S} &:= \langle -I, S \rangle.
\end{split}
\end{equation}
\begin{proposition} \label{levelincreaseboundprop}
Let $G \subseteq \GL_2(\hat{\mbz})$ be an open subgroup.  We then have
\begin{equation} \label{Gversion}
\frac{\level_{\SL_2}(G)}{\level_{\SL_2}(\tilde{G})} \in \{ 1, 2 \},
\end{equation}
where $\tilde{G}$ is as in \eqref{defoofGtildeandStilde}.
\end{proposition}
\begin{proof}
See \cite[Lemma 3.1]{jonesmcmurdy}.
\end{proof}


In the next lemma, for an open subgroup $G \subseteq \GL_2(\hat{\mbz})$ of $\GL_2$-level $m_G$ and $\SL_2$-level $m_S$, and for an arbitrary positive integer $m$ with $m_S \mid m \mid m_G$, we let $\pi_{\GL_2}$ and $\pi_{\mb{G}_m}$ denote the canonical projection maps
\begin{equation} \label{defofpi}
\begin{split}
&\pi_{\GL_2} :  \GL_2(\mbz/m_G\mbz) \longrightarrow \GL_2(\mbz/m\mbz), \\
&\pi_{\mb{G}_m} : (\mbz/m_G\mbz)^\times \longrightarrow (\mbz/m\mbz)^\times.
\end{split}
\end{equation}
\begin{lemma} \label{verticalSL2liftlemma}
Let $G \subseteq \GL_2(\hat{\mbz})$ be an open subgroup satisfying $m_G \mid m_S^\infty$ and let $m$ be any positive integer satisfying $m_S \mid m$ and $m \mid m_G$.  Then there exists a unique group homomorphism
\[
\gd : G(m) \longrightarrow (\mbz/m_G\mbz)^\times
\]
satisfying $\pi_{\mb{G}_m} \circ \gd = \det$ (where $\pi_{\mb{G}_m}$ is the canonical projection as in \eqref{defofpi}), and such that
\[
G(m_G) = \left\{ g \in \pi_{\GL_2}^{-1}\left( G(m) \right) : \gd \left( \pi_{\GL_2}(g) \right) = \det g \right\}.
\]
If $\det G = \hat{\mbz}^\times$, then $\gd$ is surjective and $\gd\left( G(m) \cap \SL_2(\mbz/m\mbz) \right) = \ker \pi_{\mb{G}_m}$.
\end{lemma}
\begin{proof}
Let $\pi_G : G(m_G) \longrightarrow G(m)$ denote the restriction to $G(m_G)$ of $\pi_{\GL_2}$.  We will first establish that
\begin{equation} \label{kerpiGequalsSL2kernel}
\ker \pi_G = \ker \left( \SL_2(\mbz/m_G\mbz) \rightarrow \SL_2(\mbz/m\mbz) \right).
\end{equation}
First, by definition of $m_S$ and since $m_S \mid m$, we have
\[
\ker \pi_G \supseteq \ker \left(  \SL_2(\mbz/m_G\mbz) \rightarrow \SL_2(\mbz/m\mbz) \right),
\]
and we will now argue by induction that these kernels have the same size.  Let $p$ be any prime dividing $m_G/m$ and factor $\pi_G$ as
\[
\begin{tikzcd}
G(m_G) \rar{\pi_p} \arrow[black, bend left]{rr}{\pi_G} & G(m_G/p) \rar{\pi_{m/p}} & G(m).
\end{tikzcd}
\]
By induction, we have that
\[
\left| \ker \pi_{m/p} \right| = \left| \ker \left(  \SL_2(\mbz/(m_G/p)\mbz) \rightarrow \SL_2(\mbz/m\mbz) \right) \right|.
\]
Since $m_G$ divides $m_S^\infty$, we see that $p$ divides $m_G/p$, and so
\[
\begin{split}
&\ker \left( \GL_2(\mbz/m_G\mbz) \rightarrow \GL_2(\mbz/(m_G/p)\mbz) \right), \\
&\ker \left( \SL_2(\mbz/m_G\mbz) \rightarrow \SL_2(\mbz/(m_G/p)\mbz) \right)
\end{split}
\]
are abelian groups of orders $p^4$ and $p^3$, respectively.  Since $m_G/p$ is not the $\GL_2$-level of $G$, we have
\[
\ker \left( \SL_2(\mbz/m_G\mbz) \rightarrow \SL_2(\mbz/(m_G/p)\mbz) \right) \subseteq \ker \pi_p \subsetneq \ker \left( \GL_2(\mbz/m_G\mbz) \rightarrow \GL_2(\mbz/(m_G/p)\mbz) \right).
\]
It follows that $\ker \left( \SL_2(\mbz/m_G\mbz) \rightarrow \SL_2(\mbz/(m_G/p)\mbz) \right) = \ker \pi_p$, so 
\[
\left| \ker \pi_G \right| = \left| \pi_p^{-1} \left( \ker \pi_{m/p} \right) \right| = \left| \ker \left( \SL_2(\mbz/m_G\mbz) \rightarrow \SL_2(\mbz/m\mbz) \right) \right|, 
\]
and \eqref{kerpiGequalsSL2kernel} is thus verified.

We now define the map $\gd : G(m) \longrightarrow (\mbz/m_G\mbz)^\times$ as follows.  For $g \in G(m)$, fix any element $\tilde{g} \in G(m_G)$ satisfying $\pi_G(\tilde{g}) = g$ and set $\gD(g) := \det \tilde{g}$.  By virtue of \eqref{kerpiGequalsSL2kernel}, we see that $\gD(g)$ is independent of the choice of lift $\tilde{g}$ and thus $\gd$ is a well-defined group homomorphism; the condition $\pi_{\mb{G}_m} \circ \gd = \det$ is immediately verified, as is
\begin{equation} \label{imageofdelta}
\gD(G(m)) = \det G(m_G).
\end{equation}
In particular, if $\det G(m_G) = (\mbz/m_G\mbz)^\times$, then $\gd$ is surjective.  Furthermore, it follows from $\pi_{\mb{G}_m} \circ \gd = \det$ that
\[
\gd\left( G(m) \cap \SL_2(\mbz/m\mbz) \right) = \ker \pi_{\mb{G}_m} \cap \gd \left( G(m) \right) .
\]
Thus, if $\gd$ is surjective then $\gd\left( G(m) \cap \SL_2(\mbz/m\mbz) \right) = \ker \pi_{\mb{G}_m}$.  Finally, 
we clearly have
\[
G(m_G) \subseteq \left\{ g \in \pi_{\GL_2}^{-1}\left( G(m) \right) : \gd\left( \pi_{\GL_2}(g) \right) = \det g \right\},
\]
and, from \eqref{kerpiGequalsSL2kernel}, the two groups are seen to have equal size, and are thus equal.
\end{proof}