\section{Bounding the $\GL_2$-level of missing trace groups of genus zero} \label{proofofboundingthelevelsthmsection}

In this section we prove Theorem \ref{boundingthelevelsthm}.  To begin with, the main results in \cite{cumminspauli} immediately imply that
\begin{equation} \label{cumminspaulilevels}
\left\{ \level_{\SL_2}(\tilde{G}) : G \in \mf{G}(0) \right\} = \left\{ \begin{matrix} 1, 2, 3, 4, 5, 6, 7, 8, 9, 10, 11, 12, 13, 14, 15, 16 \\ 18, 20, 21, 24, 25, 26, 27, 28, 30, 32, 36, 48 \end{matrix} \right\}.
\end{equation}
By Proposition \ref{levelincreaseboundprop} we then have
\begin{equation} \label{boundforSL2levels}
G \in \mf{G}(0) \; \Longrightarrow \; \level_{\SL_2}(G) \in \left\{ \begin{matrix} 1, 2, 3, 4, 5, 6, 7, 8, 9, 10, 11, 12, 13, 14, 15, 16, 18, 20, 21, 22, 24, \\ 25, 26, 27, 28, 30, 32, 36, 40, 42, 48, 50, 52, 54, 56, 60, 64, 72, 96 \end{matrix} \right\},
\end{equation}
we will now establish the following proposition, which bounds $m_G$ for $G \in \mf{G}_{MT}^{\max}(0)$ in terms of $m_S := \level_{\SL_2}(G)$.  For an open subgroup $G \subseteq \GL_2(\hat{\mbz})$, we set 
\begin{equation} \label{defofksubG}
d_G := \gcd \left( m_S^ \infty, \left| \frac{G(m_S) \cap \SL_2(\mbz/m_S\mbz)}{[G(m_S),G(m_S)]} \right| \right),
\end{equation}
i.e. $d_G$ is the largest factor of $\left| \frac{G(m_S) \cap \SL_2(\mbz/m_S\mbz)}{[G(m_S),G(m_S)]} \right|$ supported on primes dividing $m_S$.
\begin{proposition} \label{twotothekproposition}
Let $G \in \mf{G}_{MT}^{\max}$ be a maximal missing trace group of $\GL_2$-level $m_G$ and $\SL_2$-level $m_S$ satisfying $\det G = \hat{\mbz}^\times$.  Then $m_G$ divides $d_G m_S$, where $d_G$ is defined by \eqref{defofksubG}.  .
\end{proposition}
\begin{proof}
Without loss of generality, we may assume that $m_G > m_S$; we let $p$ be any prime for which $v_p(m_G) > v_p\left( m_S \right)$ and set $m_G' := m_G/p^{v_p(m_G) - v_p(m_S)}$.  Note that, for any prime $\ell$, we have
\begin{equation} \label{vpofmsubGequalsvpofmsubS}
v_{\ell}(m_G') = 
\begin{cases}
v_{\ell}(m_G) & \text{ if } \ell \neq p \\
v_{\ell}(m_S) & \text{ if } \ell = p. 
\end{cases} 
\end{equation}
In particular, since $m_S$ divides $m_G'$, we then have
\begin{equation} \label{kernelofSL2iscontainedinG}
\ker \left( \SL_2(\mbz/m_G\mbz) \rightarrow \SL_2(\mbz/m_G'\mbz) \right) \subseteq G(m_G).
\end{equation}
Furthermore, since $G$ is maximal among missing trace groups, it follows that
\begin{equation} \label{tracemodmsubGprimeissurjective}
\tr\left( G(m_G') \right) = \mbz/m_G'\mbz.
\end{equation}
We first claim that 
\begin{equation} \label{pdividesmsubGprime}
p \mid m_G'.
\end{equation}
To see this, suppose for the sake of contradiction that $p \nmid m_G'$, and define $\ga := v_p(m_G) - v_p(m_S)$.  Then, since $m_S$ divides $m_G'$, under the Chinese Remainder Isomorphism $\mbz/m_G\mbz \simeq \mbz/m_G'\mbz \times \mbz/p^{\ga}\mbz$, condition \eqref{kernelofSL2iscontainedinG} reads 
\begin{equation} \label{SL2crossoneiscontained}
\{ 1 \} \times \SL_2(\mbz/p^{\ga}\mbz) \subseteq G(m_G).
\end{equation}
By surjectivity of $\det : G(p^{\ga}) \twoheadrightarrow (\mbz/p^{\ga}\mbz)^\times$ and Goursat's lemma, we then have
\begin{equation} \label{GofmsubGasfiberedproduct}
G(m_G) \simeq G(m_G') \times_{\psi} \GL_2(\mbz/p^{\ga}\mbz),
\end{equation}
where $\psi_{m_G'} : G(m_G') \longrightarrow \Gamma$ and $\psi_p : \GL_2(\mbz/p^{\ga}\mbz) \longrightarrow \Gamma$ denote the surjective group homomorphisms onto the common quotient group $\Gamma$ implicit in the fibered product.
It follows from \eqref{SL2crossoneiscontained} that $\SL_2(\mbz/p^{\ga}\mbz) \subseteq \ker \psi_p$, and thus, for every $\gamma \in \Gamma$, there exists $d \in (\mbz/p^{\ga}\mbz)^\times$ for which
\[
\{ g \in \GL_2(\mbz/p\mbz) : \det g = d \} \subseteq \psi_p^{-1}(\gamma).
\]
Since $\tr\left( \{ g \in \GL_2(\mbz/p\mbz) : \det g = d \}  \right) = \mbz/p^{\ga}\mbz$, it is then easy to deduce from \eqref{GofmsubGasfiberedproduct} and \eqref{tracemodmsubGprimeissurjective} that $\tr\left( G(m_G) \right) = \mbz/m_G\mbz$, a contradiction.  Therefore we have established \eqref{pdividesmsubGprime}, and, by \eqref{vpofmsubGequalsvpofmsubS}, $p \mid m_S$.  Since the prime $p$ was arbitrary, it follows that
\begin{equation} \label{msubGdividesmsubStotheinfty}
m_G \mid m_S^\infty.
\end{equation}

We now apply Lemma \ref{verticalSL2liftlemma}, which asserts that there is a surjective group homomorphism $\gd : G(m_S) \longrightarrow (\mbz/m_G\mbz)^\times$ satisfying $\pi \circ \gd = \det : G(m_S) \longrightarrow (\mbz/m_S\mbz)^\times$, and for which
\[
G(m_G) = \left\{ g \in \pi_{\GL_2}^{-1}\left( G(m_S) \right) : \gd\left( \pi(g) \right) = \det g \right\}.
\]
By \eqref{msubGdividesmsubStotheinfty} and Lemma \ref{verticalSL2liftlemma}, we have that
\[
m_G/m_S = \left| \ker \left( (\mbz/m_G\mbz)^\times \rightarrow (\mbz/m_S\mbz)^\times \right) \right| = \left| \gd\left( G(m_S) \cap \SL_2(\mbz/m_S\mbz) \right) \right|,
\]
which in turn divides
\[
\left| \frac{G(m_S) \cap \SL_2(\mbz/m_S\mbz)}{[G(m_S),G(m_S)]} \right|.
\]
By \eqref{msubGdividesmsubStotheinfty}, $m_G/m_S$ also divides $m_S^\infty$, and Proposition \ref{twotothekproposition} follows.
\end{proof}
Theorem \ref{boundingthelevelsthm} follows from \eqref{boundforSL2levels} and Proposition \ref{twotothekproposition}, together with a computer computation.  The latter was carried out using the computational package MAGMA \cite{MAGMA}.


%%%%%%%%%%%%%%%%%%%%%%%%%%%%%%%%%%%%%%%%%%%%%%%%%%%%%%%%%%%%%%%%%%%%%%%%%%%%%%%%%%%%%%%%%%%%%%%%%%%%%%%%%%%%%%%%%%%%%%%%%%%%%%%%%%%%%%%%%%%%%%%%%%%%%%
%%%%%%%%%%%%%%%%%%%%%%%%%%%%%%%%%%%%%%%%%%%%%%%%%%%%%%%%%%%%%%%%%%%%%%%%%%%%%%%%%%%%%%%%%%%%%%%%%%%%%%%%%%%%%%%%%%%%%%%%%%%%%%%%%%%%%%%%%%%%%%%%%%%%%%
%%%%%%%%%%%%%%%%%%%%%%%%%%%%%%%%%%%%%%%%%%%%%%%%%%%%%%%%%%%%%%%%%%%%%%%%%%%%%%%%%%%%%%%%%%%%%%%%%%%%%%%%%%%%%%%%%%%%%%%%%%%%%%%%%%%%%%%%%%%%%%%%%%%%%%
%%%%%%%%%%%%%%%%%%%%%%%%%%%%%%%%%%%%%%%%%%%%%%%%%%%%%%%%%%%%%%%%%%%%%%%%%%%%%%%%%%%%%%%%%%%%%%%%%%%%%%%%%%%%%%%%%%%%%%%%%%%%%%%%%%%%%%%%%%%%%%%%%%%%%%


\section{Preliminaries on division fields of elliptic curves} \label{preliminariesondivisionfieldssection}

In this section, we gather various preliminary results to be used in the following section to develop explicit models for the modular curves associated to the groups occurring in Theorem \ref{boundingthelevelsthm}.  We recall the general set-up:  $G \subseteq \GL_2(\hat{\mbz})$ is an open subgroup, $\tilde{G} := \langle G, -I \rangle$ and $X_{\tilde{G}}$ is the modular curve associated to $\tilde{G}$.  Denoting by $m$ the $\GL_2$-level of $\tilde{G}$ and by $j_{\tilde{G}} : X_{\tilde{G}} \longrightarrow X(1) \simeq \mbp^1$ the forgetful map, we have that, for any $j \in \mbq - \{ 0, 1728 \}$, 
\begin{equation} \label{interpofrationalpointsonXsubG}
j \in j_{\tilde{G}}\left( X_{\tilde{G}}(\mbq) \right) \; \Longleftrightarrow \; \exists E / \mbq \; \text{ with } \;\rho_E(G_\mbq) \dot\subseteq \tilde{G} \; \text{ and } \; j_E = j.
\end{equation}
We will use repeatedly the following consequence of the Weil pairing on an elliptic curve.  Let $K$ be a field, let $E$ an elliptic curve defined over $K$, let $m \in \mbn$ be a positive integer co-prime with $\Char K$ and let
\[
\rho_{E,m} : G_K \longrightarrow \aut(E[m]) \simeq  \GL_2(\mbz/m\mbz)
\]
be the Galois representation defined by letting $G_K$ act on the $m$-torsion of $E$.  On the other hand, let
\[
\chi_m : G_K \longrightarrow \aut(\mu_m) \simeq \left( \mbz/m\mbz \right)^\times
\]
be the mod $m$ cyclotomic character, defined by letting $G_K$ act on $\mu_m$.
\begin{lemma} \label{weilpairinglemma}
We have $\det \circ \rho_{E,m} = \chi_m$.
\end{lemma}
\begin{proof}
This follows from properties of the Weil pairing; see \cite[Ch III, \S 8]{silverman}.
\end{proof}
We remark that, whenever the level $m$ of a maximal missing trace group may be written in the form $m = m_1 m_2$ with $m_1, m_2 > 1$ and $\gcd(m_1,m_2) = 1$, the group $G(m)$ decomposes via the Chinese Remainder Theorem as a fibered product
\[
G(m) = G(m_1m_2) \simeq G(m_1) \times_{\psi} G(m_2) := \{ (g_1, g_2) \in G(m_1) \times G(m_2) : \; \psi_1(g_1) = \psi_2(g_2) \},
\]
where $\psi_i : G(m_i) \longrightarrow H$ are each surjective homomorphisms onto a common non-trivial quotient group $H$ (if $H$ were trivial, then $G(m)$ would decompose as a full cartesian product, implying that either $G(m_1)$ or $G(m_2)$ would have a missing trace, contradicting maximality of $G$).  The following lemma will be useful for determining when $\rho_{E,m_1m_2}(G_\mbq) \, \dot\subseteq \, G(m_1) \times_\psi G(m_2)$.  In general, let $\mc{G}_1$ and $\mc{G}_2$ be finite groups and let 
\[
G_i, G_i' \subseteq \mc{G}_i \quad\quad \left( i \in \{1, 2 \} \right)
\]
be subgroups.  Let
\[
\psi_i : G_i \longrightarrow H, \quad \psi_i' : G_i' \longrightarrow H' \quad\quad \left( i \in \{1, 2 \} \right)
\]
be surjective homomorphisms onto the finite groups $H$ and $H'$, respectively, and denote by $G_1 \times_\psi G_2$ and $G_1' \times_{\psi'} G_2'$ the corresponding fibered products.  In particular, these fibered products are \emph{honest} in the sense that, for each $i \in \{1, 2 \}$, the canonical projections $G_1 \times_\psi G_2 \longrightarrow G_i$ and $G_1' \times_{\psi'} G_2' \longrightarrow G_i'$ are each surjective.
\begin{lemma} \label{whenisgroupinsidefiberedproduct}
With the notation just outlined, we have
\[
G_1' \times_{\psi'} G_2' \subseteq G_1 \times_{\psi} G_2
\]
if and only if
\begin{enumerate}
\item $\forall i \in \{ 1, 2 \}, \, G_i' \subseteq G_i$,
\item there exists a group homomorphism $\varpi : H' \longrightarrow H$ such that $\forall i \in \{1, 2 \}, \; \varpi \circ \psi_i' = \psi_i \vert_{G_i'}$.
\end{enumerate}
\end{lemma}
\begin{proof}
For the direction ``$\Rightarrow$'', condition (1) is immediate.  To verify condition (2), we first note that $\ker \psi_1' \times \ker \psi_2' \subseteq G_1 \times_\psi G_2$, which implies that
\[
\ker \psi_i' \subseteq \ker \psi_i \quad\quad \left( i \in \{1, 2 \} \right).
\]
We thus have a well-defined group homomorphism $\varpi : H' \longrightarrow H$, given by $\varpi \left( \psi_1'(g_1') \right) := \psi_1(g_1')$.
We observe that $\varpi \circ \psi_i' = \psi_i \vert_{G_i'}$ by definition when $i = 1$ and by $G_1' \times_{\psi'} G_2' \subseteq G_1 \times_{\psi} G_2$ when $i = 2$.  This establishes (2).

For the converse, assume that (1) and (2) hold and let $(g_1', g_2') \in G_1' \times_{\psi'} G_2'$.  By (1), $g_1' \in G_1$ and $g_2' \in G_2$.  Furthermore, $\psi_1'(g_1') = \psi_2'(g_2')$, and thus
\[
\psi_1(g_1') = \varpi \left( \psi_1'(g_1') \right) = \varpi \left( \psi_2'(g_2') \right) = \psi_2(g_2').
\]
Thus, $(g_1',g_2') \in G_1 \times_\psi G_2$, and we have proved the converse, establishing the lemma.
\end{proof}


By \eqref{interpofrationalpointsonXsubG}, we need only consider such groups up to conjugation inside $\GL_2(\mbz/m\mbz)$, i.e. up to the relation $\doteq$.  It is straightforward to see that, for any inner automorphisms $\eta_1, \eta_2 : H \longrightarrow H$, we have
\[
G(m_1) \times_{(\eta_1\psi_1, \eta_2 \psi_2)} G(m_2) \doteq G(m_1) \times_{(\psi_1, \psi_2)} G(m_2).
\]
However, the same is not always true for outer automorphisms $\eta_1, \eta_2 \in \aut(H)$; given such a pair $(\eta_1,\eta_2) \in \aut(H)^2$, we clearly have
\[
G(m_1) \times_{(\eta_1 \psi_1, \eta_2 \psi_2)} G(m_2) = G(m_1) \times_{(\psi_1, \eta_1^{-1} \eta_2 \psi_2)} G(m_2).
\]
Thus, it suffices to consider postcomposing the pair $(\psi_1,\psi_2)$ with pairs of automorphsims of the form $(1,\eta_2)$, or of the form $(\eta_1,1)$.  
\begin{Definition} \label{defofGL2induced}
Given the notation above, we say that an automorphism $\eta_i \in \aut(H)$ \emph{\textbf{is $\GL_2(\mbz/m_i\mbz)$-induced}} if there exists $g_i \in \GL_2(\mbz/m_i\mbz)$ satisfying $g_i G(m_i) g_i^{-1} = G(m_i)$ and for which the diagram
\[
\begin{tikzcd}
G(m_i) \rar{\conj_{g_i}} \dar{\psi_i} & G(m_i) \dar{\psi_i} \\
H \rar{\eta_i} & H
\end{tikzcd}
\]
commutes.  
\end{Definition}
The following useful lemma is straightforward to prove.
\begin{lemma} \label{GL2inducedlemma}
With notation as just outlined and with $\eta_2 \in \aut(H)$, we have
\[
\begin{split}
&G(m_1) \times_{(\psi_1,\eta_2\psi_2)} G(m_2) \doteq G(m_1) \times_{(\psi_1,\psi_2)} G(m_2) \; \Longleftrightarrow \; \eta_2 \in \aut(H) \text{ is $\GL_2(\mbz/m_2\mbz)$-induced,} \\
&G(m_1) \times_{(\eta_1\psi_1,\psi_2)} G(m_2) \doteq G(m_1) \times_{(\psi_1,\psi_2)} G(m_2) \; \Longleftrightarrow \; \eta_1 \in \aut(H) \text{ is $\GL_2(\mbz/m_1\mbz)$-induced.} 
\end{split}
\]
\end{lemma}
We will now describe an interpretation of $\rho_{E,m_1m_2}(G_\mbq) \, \dot\subseteq \, G(m_1) \times_\psi G(m_2)$ in terms of entanglements.  It will be convenient to decompose the representation $\rho_{E}$ as
\begin{equation*}
\begin{tikzcd}
G_\mbq \rar{\tilde{\rho}_{E}} \arrow[black, bend left]{rr}{\rho_E} & \aut(E_{\tors}) \rar{\iota_{\mc{B}}} & \GL_2(\hat{\mbz}),
\end{tikzcd}
\end{equation*}
where $\mc{B} = \{ \mc{B}(m) := ({\bf{b}}_{1,m}, {\bf{b}}_{2,m}) : m \in \mbn \}$ is a collection of ordered $\mbz/m\mbz$-bases of $E[m] \subseteq E_{\tors}$, one for each $m \in \mbn$, chosen compatibly.  This decomposition has a corresponding a finite level analogue 
\begin{equation}  \label{defofrhotilde}
\rho_{E,m} = \iota_{\mc{B}(m)} \circ \tilde{\rho}_{E,m}
\end{equation}
for any $m \in \mbn$.  We will denote simply by $\iota$ either of the the isomorphisms $\aut(E[m]) \to \GL_2(\mbz/m\mbz)$ or $\aut(E_{\tors}) \to \GL_2(\hat{\mbz})$ induced by such a collection $\mc{B}$, suppressing the dependence on $\mc{B}$.  Thus, for any $m \in \mbn$, we have
\[
\rho_{E,m}(G_\mbq) \, \dot \subseteq \, G(m) \; \Longleftrightarrow \; \exists \iota : \tilde{\rho}_{E,m}(G_\mbq) \hookrightarrow G(m).
\]

Unraveling \eqref{interpofrationalpointsonXsubG}, we find that
\[
%\begin{tikzcd}
\exists \iota : \tilde{\rho}_{E,m_1m_2}(G_\mbq) \hookrightarrow G(m_1) \times_\psi G(m_2) \; \Longrightarrow \; \begin{matrix} \exists \iota_1 : \tilde{\rho}_{E,m_1}(G_\mbq) \hookrightarrow G(m_1), \; \exists \iota_2 : \tilde{\rho}_{E,m_2}(G_\mbq) \hookrightarrow G(m_2), \\ \text{and } \; \mbq(E[m_1])^{\iota_1^{-1}(\ker \psi_1)} = \mbq(E[m_2])^{\iota_2^{-1}(\ker \psi_2)},\end{matrix}
%\end{tikzcd}
\]
where we are understanding the subfield $\mbq(E[m_i])^{\iota_i^{-1}(\ker \psi_i)} \subseteq \mbq(E[m_i])$ via the natural isomorphism $\gal\left( \mbq(E[m_i])/\mbq \right) \simeq \tilde{\rho}_{E,m_i}(G_\mbq)$.
The following corollary states conditions under which the converse holds. 
\begin{cor} \label{justentanglementequalitycorollary}
Let $\aut_{\GL_2(\mbz/m_i\mbz)}(H) \subseteq \aut(H)$ denote the subgroup of $\GL_2(\mbz/m_i\mbz)$-induced automorphisms, and suppose that either $\aut_{\GL_2(\mbz/m_1\mbz)}(H) = \aut(H)$ or $\aut_{\GL_2(\mbz/m_2\mbz)}(H) = \aut(H)$.  We then have
\[
\rho_{E,m_1m_2}(G_\mbq) \, \dot\subseteq \, G(m_1) \times_\psi G(m_2) \; \Longleftrightarrow \; \begin{matrix} \exists \iota_1 : \tilde{\rho}_{E,m_1}(G_\mbq) \hookrightarrow G(m_1), \; \exists \iota_2 : \tilde{\rho}_{E,m_2}(G_\mbq) \hookrightarrow G(m_2), \\ \text{and } \quad \mbq(E[m_1])^{\iota_1^{-1}(\ker \psi_1)} = \mbq(E[m_2])^{\iota_2^{-1}(\ker \psi_2)}. \end{matrix}
\]
\end{cor}
We would like to apply Corollary \ref{justentanglementequalitycorollary} to groups $G \in \mf{G}_{MT}^{\max}(0)$.  Our computation shows that, for every such group $G$ whose $\GL_2$-level $m$ is divisible by at least two primes, and for any $m_1, m_2 > 1$ with $m = m_1m_2$ and $\gcd(m_1,m_2) = 1$, writing $G(m) \simeq G(m_1) \times_{\psi} G(m_2)$ and denoting by $H := \psi_i(G(m_i))$ the common quotient group implicit in the fibered product, we have
\begin{equation} \label{formsofH}
H \in \{ \mbz/2\mbz, \mbz/3\mbz, \mbz/6\mbz, S_3 \}.
\end{equation}
When $H \in \{ \mbz/2\mbz, S_3 \}$, all automorphisms of $H$ are inner, whence $\GL_2(\mbz/m_i\mbz)$-inner (no matter what the level $m_i$ is.  For the case $H \in \{ \mbz/3\mbz, \mbz/6\mbz \}$,  we will now look in more detail at the particulars.  

The common quotient $H = \mbz/3\mbz$ arises as an entanglement between the groups $G(2)$ and $G(7)$, and in all such cases, we have
\begin{equation} \label{descriptionofpsisub2in14case}
G(2) = \left\langle \begin{pmatrix} 1 & 1 \\ 1 & 0 \end{pmatrix} \right\rangle \subseteq \GL_2(\mbz/2\mbz), \quad\quad \psi_2 :  \left\langle \begin{pmatrix} 1 & 1 \\ 1 & 0 \end{pmatrix} \right\rangle \simeq \mbz/3\mbz.
\end{equation}
Thus, the image $H = \mbz/3\mbz$ is isomorphic to $G(2)$, an index two (and hence normal) subgroup of $\GL_2(\mbz/2\mbz)$.  Since
\[
\begin{pmatrix} 0 & 1 \\ 1 & 0 \end{pmatrix} \begin{pmatrix} 1 & 1 \\ 1 & 0 \end{pmatrix} \begin{pmatrix} 0 & 1 \\ 1 & 0 \end{pmatrix}^{-1} = \begin{pmatrix} 0 & 1 \\ 1 & 1 \end{pmatrix},
\]
it follows that the conjugation action of $\GL_2(\mbz/2\mbz)$ on $G(2)$ gives rise to all automorphisms of $H$, i.e. that $\aut(H) = \aut_{\GL_2(\mbz/2\mbz)}(H)$ in this case.

The common quotient $H \simeq \mbz/6\mbz$ arises as an entanglement between $G(4)$ and $G(7)$, and in all such cases, we have
\[
G(4) = \pi_{\GL_2}^{-1}\left( \left\langle \begin{pmatrix} 1 & 1 \\ 1 & 0 \end{pmatrix} \right\rangle \right) \subseteq \GL_2(\mbz/4\mbz), \quad\quad \ker \psi_4 = \ker \pi_{\GL_2} \cap \SL_2(\mbz/4\mbz),
\]
where $\pi_{\GL_2} : \GL_2(\mbz/4\mbz) \to \GL_2(\mbz/2\mbz)$ denotes the canonical projection map.  Therefore the map $\psi_4$ decomposes as
\[
\begin{tikzcd}
\pi_{\GL_2}^{-1}\left( \left\langle \begin{pmatrix} 1 & 1 \\ 1 & 0 \end{pmatrix} \right\rangle \right) \rar{\psi_2 \times \det} \arrow[black, bend left]{rr}{\psi_4} & \mbz/3\mbz \times \mbz/2\mbz \rar{\simeq} & \mbz/6\mbz,
\end{tikzcd}
\]
where $\psi_2$ denotes the function $g \mapsto \psi_2(g \bmod 2)$, with $\psi_2$ as in \eqref{descriptionofpsisub2in14case}.  Thus, it follows from the discussion in the previous paragraph that $\aut_{\GL_2(\mbz/4\mbz)}(H) = \aut(H)$ in this case as well.  Taking these observations together with the computation that establishes \eqref{formsofH}, we thus have
\begin{cor} \label{keycorollaryforinterpretationofentanglements}
For each group $G \in \mf{G}_{MT}^{\max}(0)$ with the property that $m := \level_{\GL_2}(G)$ is divisible by at least two primes, choose any $m_1, m_2 \in \mbn$ with $\gcd(m_1,m_2) = 1$ and $m_1, m_2 > 1$ and write $G(m) \simeq G(m_1) \times_{\psi} G(m_2)$.  For any elliptic curve $E$ over $\mbq$, we have
\[
\rho_{E,m_1m_2}(G_\mbq) \, \dot\subseteq \, G(m_1) \times_\psi G(m_2) \; \Longleftrightarrow \; \begin{matrix} \exists \iota_1 : \tilde{\rho}_{E,m_1}(G_\mbq) \hookrightarrow G(m_1), \; \exists \iota_2 : \tilde{\rho}_{E,m_2}(G_\mbq) \hookrightarrow G(m_2), \\ \text{and } \quad \mbq(E[m_1])^{\iota_1^{-1}(\ker \psi_1)} = \mbq(E[m_2])^{\iota_2^{-1}(\ker \psi_2)},
\end{matrix}
\]
where the basis-induced embeddings $\iota_1$ and $\iota_2$ and the representations $\tilde{\rho}_{E,m_i}$ are as in \eqref{defofrhotilde}, and the subfield $\mbq(E[m_i])^{\iota_i^{-1}(\ker \psi_i)} \subseteq \mbq(E[m_i])$ is understood via the natural isomorphism $\tilde{\rho}_{E,m_i}(G_\mbq) \simeq \gal\left( \mbq(E[m_i])/\mbq \right)$.
\end{cor}


\subsection{Cyclic cubic fields}

Since we will be dealing with cyclic cubic extensions of $\mbq(t,D)$, we now state two lemmas about such field extensions that will be used in what follows.
Our first lemma allows us to exhibits explicit polynomials for generators of each of the cyclic cubic  subfields of a given $\mbz/3\mbz \times \mbz/3\mbz$-extension.  In general, let $K$ be any field, let
\begin{equation} \label{defoffsubSandfsubT}
\begin{split}
f_S(x) &= x^3 - S_1x^2 + S_2 x - S_3, \\
f_T(x) &= x^3 - T_1x^2 + T_2 x - T_3
\end{split}
\end{equation}
be two monic irreducible polynomials with coefficients in $K$, and denote by $K_{f_S}$ (resp. by $K_{f_T}$) the splitting polynomial of $f_S$ (resp. of $f_T$), viewed as subfields of a fixed algebraic closure $\ol{K}$ of $K$.  Assume that 
\begin{equation} \label{KfsubSdifferentfromKfsubT}
K_{f_S} \neq K_{f_T}
\end{equation}
and that the discriminants $\gD_S := \disc(f_S)$ and $\gD_T := \disc(f_T)$ are each in $\left( K^\times \right)^2$, or equivalently that $\gal(K_{f_S}/K) \simeq \gal(K_{f_T}/K)$ is a cyclic group of order $3$.  The assumption \eqref{KfsubSdifferentfromKfsubT} then implies that the composite field $K_{f_Sf_T} = K_{f_S}K_{f_T}$ satisfies
\begin{equation} \label{galoisgroupisZmod3timesZmod3}
\gal(K_{f_Sf_T}/K) \simeq \mbz/3\mbz \times \mbz/3\mbz,
\end{equation}
and this field contains $4$ cyclic cubic subfields.  Fix square roots $\sqrt{\gD_S}, \sqrt{\gD_T} \in K$ and define the coefficients $R_1, R_2, R_3, R_3' \in K$ by
\begin{equation} \label{defofRsubis}
\begin{split}
R_1 := &S_1 T_1, \\
R_2 := &S_1^2 T_2 + T_1^2 S_2 - 3S_2T_2, \\
R_3 := &S_1^3 T_3 + T_1^3 S_3 - 3S_1 S_2 T_3 - 3 T_1 T_2 S_3 + 9S_3 T_3 \\
&+ \left( S_1S_2 - 3S_3 + \sqrt{\gD_S} \right) \left( T_1 T_2 - 3T_3 + \sqrt{\gD_T} \right) / 4 \\
&+ \left( S_1S_2 - 3S_3 - \sqrt{\gD_S} \right) \left( T_1 T_2 - 3T_3 - \sqrt{\gD_T} \right) / 4, \\
R_3' := &S_1^3 T_3 + T_1^3 S_3 - 3S_1 S_2 T_3 - 3 T_1 T_2 S_3 + 9S_3 T_3 \\
&+ \left( S_1S_2 - 3S_3 + \sqrt{\gD_S} \right) \left( T_1 T_2 - 3T_3 - \sqrt{\gD_T} \right) / 4 \\
&+ \left( S_1S_2 - 3S_3 - \sqrt{\gD_S} \right) \left( T_1 T_2 - 3T_3 + \sqrt{\gD_T} \right) / 4.
\end{split}
\end{equation}
Define the cubic polynomials $f_R(x), f_{R'}(x) \in K[x]$ by 
\begin{equation} \label{defoffsubRandfsubRprime}
\begin{split}
f_R(x) &:= x^3 - R_1x^2 + R_2x - R_3, \\
f_{R'}(x) &:= x^3 - R_1x^2 + R_2x - R_3'.
\end{split}
\end{equation}
\begin{lemma} \label{gettingattheothercubicfieldslemma}
Let $K$ be a field, let $f_S(x), f_T(x) \in K[x]$ be irreducible monic cubic polynomials as in \eqref{defoffsubSandfsubT}, and assume the setup and notation laid out above (in particular, assume that the splitting field $K_{f_Sf_T}$ of $f_S(x)f_T(x)$ satisfies \eqref{galoisgroupisZmod3timesZmod3}).  Then the four cyclic cubic subfields of $K_{f_Sf_T}$ are the splitting fields of the polynomials $f_S(x)$, $f_T(x)$, $f_R(x)$ and $f_{R'}(x)$, where $f_R(x)$ and $f_{R'}(x)$ are defined by \eqref{defoffsubRandfsubRprime} and \eqref{defofRsubis}.
\end{lemma}
\begin{proof}
An exercise in symmetric polynomials.
\end{proof}

Given a field $K$ and elements $S_1, S_2, S_3, T_1, T_2, T_3 \in K$, we may consider the following system of equations in the variables $a$, $b$ and $c$.
\begin{equation} \label{abcequations}
\begin{split}
T_1 = &a \left( S_1^2 - 2S_2 \right) + b S_1 + 3c, \\
T_2 = &a^2 S_2^2 - 2a^2 S_1 S_3 + ab S_1 S_2 - 3abS_3 + 2acS_1^2 - 4acS_2 + b^2 S_2 + 2bc S_1 + 3c^2, \\
T_3 = &a^3 S_3^2 + a^2b S_2 S_3 - 2a^2c S_1 S_3 + a^2c S_2^2 + ab^2 S_1 S_3 + abc S_1 S_2 - 3abc S_3 \\
&+ ac^2 S_1^2 - 2ac^2 S_2 + b^3 S_3 + b^2 c S_2 + bc^2 S_1 + c^3.
\end{split}
\end{equation}

\begin{lemma} \label{settingcubicfieldsequaltoeachotherlemma}
Let $K$ be a field, let $f_S(x)$, $f_T(x) \in K[x]$ be irreducible monic cubic polynomials as in \eqref{defoffsubSandfsubT} and denote by $K_{f_S}$ (resp. by $K_{f_T}$) the splitting field of $f_S(x)$ (resp. of $f_T(x)$), viewed as subfields of a fixed algebraic closure $\ol{K}$ of $K$.  Assume that the discriminant $\gD_S$ of $f_S(x)$ satisfies $\gD_S \in \left( K^\times \right)^2$, so that $\gal(K_{f_S}/K)$ is a cyclic group of order $3$. We then have that $K_{f_S} = K_{f_T}$ if and only if the system of equations \eqref{abcequations} has a solution $(a,b,c) \in K^3$.
\end{lemma}
\begin{proof}
Let $\ga \in \ol{K}$ denote a root of $f_S(x)$.  Then $K_{f_S} = K(\ga)$, and so we may write an arbitrary element $\gb \in K_{f_S}$ in the form
\begin{equation} \label{betaasalgebraicexpressioninalpha}
\gb = a \ga^2 + b\ga + c \quad\quad \left( a, b, c \in K \right).
\end{equation}
The elementary symmetric polynomials $S_i(a \ga^2 + b\ga + c)$ of such an element are then readily computed to be
\begin{equation} \label{symmetricpolynomialequalities}
\begin{split}
S_1(a \ga^2 + b\ga + c) = &a \left( S_1^2 - 2S_2 \right) + b S_1 + 3c, \\
S_2(a \ga^2 + b\ga + c) = &a^2 S_2^2 - 2a^2 S_1 S_3 + ab S_1 S_2 - 3abS_3 + 2acS_1^2 - 4acS_2 + b^2 S_2 + 2bc S_1 + 3c^2, \\
S_3(a \ga^2 + b\ga + c) = &a^3 S_3^2 + a^2b S_2 S_3 - 2a^2c S_1 S_3 + a^2c S_2^2 + ab^2 S_1 S_3 + abc S_1 S_2 - 3abc S_3 \\
&+ ac^2 S_1^2 - 2ac^2 S_2 + b^3 S_3 + b^2 c S_2 + bc^2 S_1 + c^3.
\end{split}
\end{equation}
Thus, if \eqref{abcequations} have a solution $(a,b,c) \in K^3$, we see that $f_T(x)$ has a root in $K_{f_S}$, thus $f_T(x)$ splits completely over $K_{f_S}$, and so $K_{f_T} = K_{f_S}$.  Conversely, if $K_{f_T} = K_{f_S}$, then let $\gb \in K_{f_T}$ be a root of $f_T(x)$.  Writing $\gb$ in the form \eqref{betaasalgebraicexpressioninalpha}, we see that $(a,b,c) \in K^3$ is then a solution to \eqref{abcequations}.
\end{proof}

\subsection{Division fields of elliptic curves and their subfields} \label{divisionfieldssubsection}

In this section, we exhibit explicitly various subfields of the $m$th division field $\mbq(t,D)\left( \mc{E}[m] \right)$ for various levels $m$ and elliptic curves $\mc{E}$ over $\mbq(t,D)$.  The Borel subgroup
\[
B(\ell) := \left\{ \begin{pmatrix} * & * \\ 0 & * \end{pmatrix} \right\} \subseteq \GL_2(\mbz/\ell\mbz)
\]
plays a key role, as do the two multiplicative homomorphisms $\psi_{\ell,1}, \psi_{\ell,2} : B(\ell) \longrightarrow (\mbz/\ell\mbz)^\times$ defined by
\begin{equation} \label{generaldefinitionofpsisubp}
%\begin{split}
\psi_{\ell,1}\left( \begin{pmatrix} a & b \\ 0 & d \end{pmatrix} \right) := a, \quad\quad \psi_{\ell,2}\left( \begin{pmatrix} a & b \\ 0 & d \end{pmatrix} \right) := d.
%\end{split}
\end{equation}
For any elliptic curve $E$ over $\mbq$, we have
\begin{equation} \label{interpretationofwhenGaloismapsintoBorel}
\rho_{E,\ell}(G_\mbq) \subseteq B(\ell) \; \Longleftrightarrow \; \text{ $\exists$ a $G_\mbq$-stable cyclic subgroup } \langle P \rangle \subseteq E[\ell].
\end{equation}
When this is the case, we denote by $E_{\langle P \rangle}' := E / \langle P \rangle$ the quotient curve, which is isogenous over $\mbq$ to $E$.  We have
$
\rho_{E_{\langle P \rangle}',\ell}(G_\mbq) \subseteq B(\ell),
$
or in other words, 
\begin{equation} \label{existenceofPprime}
\text{ $\exists$ a $G_\mbq$-stable cyclic subgroup } \langle P' \rangle \subseteq E_{\langle P \rangle}'[\ell]
\end{equation}
(this is the kernel of the dual isogeny $E_{\langle P \rangle}' \rightarrow E$).  In these terms, the Galois representations $\psi_{\ell,1}$ and $\psi_{\ell,2}$ above are simply defined by restricting the action of $G_\mbq$ respectively to $\langle P \rangle$ and to $\langle P' \rangle$, i.e. we have
\begin{equation} \label{psisubellsactingonPandPprime}
\gs : P \mapsto \left[ \psi_{\ell,1}(\gs) \right] P, \quad\quad\quad \gs : P' \mapsto \left[ \psi_{\ell,2}(\gs) \right] P' \quad\quad \left( \gs \in G_\mbq \right).
\end{equation}
Finally, we note that $\psi_{\ell,i}^{(\ell-1)/2}\left( g \right) \in \{ \pm 1 \} \subseteq (\mbz/\ell\mbz)^\times$, and that this value agrees with the Legendre symbol evaluated at $\psi_{\ell,i}\left( g \right)$, i.e. we have
\[
\psi_{\ell,1}^{(\ell-1)/2}\left( \begin{pmatrix} a & b \\ 0 & d \end{pmatrix} \right) \equiv \left( \frac{a}{\ell} \right) \bmod \ell, \quad\quad \psi_{\ell,2}^{(\ell-1)/2}\left( \begin{pmatrix} a & b \\ 0 & d \end{pmatrix} \right) \equiv \left( \frac{d}{\ell} \right) \bmod \ell.
\]


\subsubsection{The level \texorpdfstring{$m = 2$}}

The group $\GL_2(\mbz/2\mbz)$ is a non-abelian group of order $6$, and since there is only one such group up to isomorphism, we see that it is isomorphic to the symmetric group of order $6$:  
\begin{equation} \label{GL2Zmod2ZisisomorphictoD3}
\GL_2(\mbz/2\mbz) \simeq S_3.
\end{equation}
As such, there is a unique proper non-trivial normal subgroup 
\[
\left\langle \begin{pmatrix} 1 & 1 \\ 1 & 0 \end{pmatrix} \right\rangle \subseteq \GL_2(\mbz/2\mbz)
\]
which has index two (corresponding under \eqref{GL2Zmod2ZisisomorphictoD3} to the alternating subgroup $A_3$).  This index two subgroup happens to be the commutator subgroup $\SL_2(\mbz/2\mbz)'$ of $\SL_2(\mbz/2\mbz)$; as we shall see, both this fact and the next classical, well-known lemma generalize to levels $3$ and $4$.
\begin{lemma} \label{level2kummersubextensionlemma}
Let $E$ be an elliptic curve over $\mbq$ and let $\gD_E$ denote the discriminant of any\footnote{Note that the field $\mbq(\sqrt{\gD_E})$ (resp. the fields $\mbq(\gD_E^{1/3})$ and $\mbq(\gD_E^{1/4})$ appearing in Lemmas \ref{level3kummersubextensionlemma} and \ref{level4kummersubextensionlemma}) does not depend on the choice of Weierstrass model for $E$, even though the discriminant $\gD_E$ does.} Weierstrass model of $E$.  We have that $\mbq(\sqrt{\gD_E}) \subseteq \mbq(E[2])$.  Furthermore, this subfield corresponds via Galois theory to the subgroup $\rho_{E,2}(G_\mbq) \cap \SL_2(\mbz/2\mbz)'$, i.e. we have
\[
\mbq(E[2])^{\rho_{E,2}(G_\mbq) \cap \SL_2(\mbz/2\mbz)'} = \mbq\left( \sqrt{\gD_E} \right).
\]
\end{lemma}
\begin{proof}
See for instance \cite[pp. 218]{langtrotter}.
\end{proof}
Throughout the paper, the role played by restriction map $\gal\left( \mbq(E[2])/\mbq \right) \rightarrow \gal\left( \mbq\left( \sqrt{\gD_E} \right)/\mbq \right)$ is significant enough to warrant our giving it an explicit name.  We make the definition
\begin{equation} \label{defofve}
\begin{tikzcd}
\ve : \GL_2(\mbz/2\mbz) \rar{\simeq}& S_3 \rar{\can}& \frac{S_3}{A_3} \rar{\simeq}& \{ \pm 1 \}.
\end{tikzcd}
\end{equation} 
Thus, we have
\[
\ker \ve = \left\langle \begin{pmatrix} 1 & 1 \\ 1 & 0 \end{pmatrix} \right\rangle \quad\quad \text{ and } \quad\quad \mbq(E[2])^{\ker \ve} = \mbq\left( \sqrt{\gD_E} \right).
\]


\subsubsection{The level \texorpdfstring{$m = 3$}}

Using the classical theory of modular functions (see \cite{zywina}), it can be shown that there is a rational parameter $t$ on the genus zero modular curve $X_0(3)$ such that the forgetful map $X_0(3) \longrightarrow X(1)$ takes the form
\[
\mbp^1(t) \longrightarrow \mbp^1(j), \quad t \mapsto 27 \frac{(t+1)(t+9)^3}{t^3}.
\]
We define $\displaystyle j_3(t) := 27 \frac{(t+1)(t+9)^3}{t^3} \in \mbq(t)$ and the elliptic curve $\mc{E}_{3}$ over $\mbq(t,D)$ by
\begin{equation} \label{level3genericellipticcurve}
\mc{E}_3 : \; y^2 = x^3 + \frac{108D^2 j_3(t)}{1728 - j_3(t)} x + \frac{432D^3 j_3(t)}{1728 - j_3(t)}.
\end{equation}
By restricting the action of $\gal\left( \mbq(t,D)(\mc{E}_3[3])/\mbq(t,D) \right)$ to $\mc{E}_3[3]$ and fixing a $\mbz/3\mbz$-basis thereof, we obtain an isomorphism
\begin{equation} \label{genericGaloisimageiscontainedinborelmod3}
\gal\left( \mbq(t,D)(\mc{E}_3[3])/\mbq(t,D) \right) \simeq \left\{ \begin{pmatrix} * & * \\ 0 & * \end{pmatrix} \right\} \subseteq \GL_2(\mbz/3\mbz).
\end{equation}
The following lemma explicitly characterizes the subfields cut out by the characters $\psi_{3,1}$ and $\psi_{3,2}$.
\begin{lemma} \label{subfieldsoflevel3lemma}
Let $\mc{E}_3$ be the elliptic curve over $\mbq(t,D)$ defined by \eqref{level3genericellipticcurve} and define the characters 
\[
\psi_{3,1}, \psi_{3,2} : \gal\left( \mbq(t,D)(\mc{E}_3[3])/\mbq(t,D) \right) \rightarrow \{ \pm 1 \}
\]
to be the restrictions under \eqref{genericGaloisimageiscontainedinborelmod3} of the characters defined in \eqref{generaldefinitionofpsisubp}.  We then have
\begin{equation} \label{subfieldsoflevel3lemmaeqn}
\begin{split}
\mbq(t,D)(\mc{E}_3[3])^{\ker \psi_{3,1}} &= \mbq(t,D)\left( \sqrt{\frac{6D(t+1)(t+9)}{(t^2-18t-27)}} \right), \\
\mbq(t,D)(\mc{E}_3[3])^{\ker \psi_{3,2}} &= \mbq(t,D)\left( \sqrt{-\frac{2D(t+1)(t+9)}{(t^2-18t-27)}} \right).
\end{split}
\end{equation}
\end{lemma}
\begin{proof}
We compute that the $3$rd division polynomial associated to $\mc{E}_3$ has the $\mbq(t,D)$-rational factor 
\[
x - \frac{18D(t+1)(t+9)}{t^2 - 18t - 27}, 
\]
and this leads us to the point
\[
P := \left( \frac{18D(t+1)(t+9)}{t^2 - 18t - 27}, \frac{24Dt(t+9)}{t^2 - 18t - 27} \sqrt{\frac{6D(t+1)(t+9)}{(t^2-18t-27)}} \right) \in \mc{E}_3[3].
\]
Since $\displaystyle \mbq(t,D)(\mc{E}_3[3])^{\ker \psi_{3,1}} = \mbq(t,D)\left( P \right)$, this establishes the first formula in \eqref{subfieldsoflevel3lemmaeqn}; the expression for the fixed field of $\ker \psi_{3,2}$ then follows from the fact that $\displaystyle \psi_{3,1}(g) \psi_{3,2}(g) = \det g$, whose corresponding fixed field is $\mbq(t,D) \left( \sqrt{-3} \right)$.
\end{proof}

We will also make use of the following classical fact about the third division field of an elliptic curve.  Note that the commutator subgroup $\SL_2(\mbz/3\mbz)' := \left[ \SL_2(\mbz/3\mbz), \SL_2(\mbz/3\mbz) \right]$ is a normal subgroup of $\GL_2(\mbz/3\mbz)$ and the quotient group is dihedral of order 6:
\[
\frac{\GL_2(\mbz/3\mbz)}{\SL_2(\mbz/3\mbz)'} \simeq D_3.
\] 
Thus, the associated fixed field $\mbq(E[3])^{\rho_{E,3}(G_\mbq) \cap \SL_2(\mbz/3\mbz)'} \subseteq \mbq(E[3])$ is generically a $D_3$-extension of $\mbq$; the next lemma specifies generators for this subfield.
\begin{lemma} \label{level3kummersubextensionlemma}
Let $E$ be an elliptic curve over $\mbq$ and let $\gD_E$ denote the discriminant of any Weierstrass model of $E$.  We have that $\mbq\left( \mu_3, \gD_E^{1/3} \right) \subseteq \mbq(E[3])$.  Furthermore, this subfield corresponds via Galois theory to the subgroup $\rho_{E,3}(G_\mbq) \cap \SL_2(\mbz/3\mbz)'$, i.e. we have
\[
\mbq(E[3])^{\rho_{E,3}(G_\mbq) \cap \SL_2(\mbz/3\mbz)' } = \mbq \left( \mu_3, \gD_E^{1/3} \right).
\]
\end{lemma}
\begin{proof}
This is a classical result; see for instance \cite[pp. 181--183]{langtrotter} and the references therein.
\end{proof}

\subsubsection{The level \texorpdfstring{$m=4$}}

The following lemma details the relevant classical facts surrounding the fourth division field of an elliptic curve.  Note that the commutator subgroup $\SL_2(\mbz/4\mbz)' := \left[ \SL_2(\mbz/4\mbz), \SL_2(\mbz/4\mbz) \right]$ is a normal subgroup of $\GL_2(\mbz/4\mbz)$ and the quotient group is dihedral of order 8:
\[
\frac{\GL_2(\mbz/4\mbz)}{\SL_2(\mbz/4\mbz)'} \simeq D_4.
\] 
Thus, the associated fixed field $\mbq(E[4])^{\rho_{E,4}(G_\mbq) \cap \SL_2(\mbz/4\mbz)'} \subseteq \mbq(E[4])$ is generically a $D_4$-extension of $\mbq$; we now specify generators for this subfield.
\begin{lemma} \label{level4kummersubextensionlemma}
Let $E$ be an elliptic curve over $\mbq$ and let $\gD_E$ denote the discriminant of any Weierstrass model of $E$.  We have that $\mbq\left( \mu_4, \gD_E^{1/4} \right) \subseteq \mbq(E[4])$.  Furthermore, this subfield corresponds via Galois theory to the subgroup $\rho_{E,4}(G_\mbq) \cap \SL_2(\mbz/4\mbz)'$, i.e. we have
\[
\mbq(E[4])^{\rho_{E,4}(G_\mbq) \cap \SL_2(\mbz/4\mbz)' } = \mbq \left( \mu_4, \gD_E^{1/4} \right).
\]
\end{lemma}
\begin{proof}
See \cite[pp. 172--173]{langtrotter} and \cite[pp. 218--220]{langtrotter}.
\end{proof}

We will sometimes need to deal with this subfield in the case that $\rho_{E,4}(G_\mbq)$ is contained in a specific proper subgroup of $\GL_2(\mbz/4\mbz)$.  In particular, we will be interested in the subgroup
\begin{equation} \label{defofGL2subchi4equalsve}
\begin{split}
\GL_2(\mbz/4\mbz)_{\chi_4 = \ve} :=& \left\{ g \in \GL_2(\mbz/4\mbz) : \chi_4(\det g) = \ve(g \bmod 2) \right\} \\
=& \left\langle \begin{pmatrix} 1 & 1 \\ 0 & 3 \end{pmatrix}, \begin{pmatrix} 1 & 0 \\ 1 & 3 \end{pmatrix}, \begin{pmatrix} 1 & 3 \\ 1 & 0 \end{pmatrix} \right\rangle,
\end{split}
\end{equation}
where $\chi_4 : (\mbz/4\mbz)^\times \rightarrow \{ \pm 1 \}$ is the unique nontrivial multiplicative character and $\ve : \GL_2(\mbz/2\mbz) \rightarrow \{ \pm 1 \}$ is as in \eqref{defofve}.  For any elliptic curve $E$ over $\mbq$, we have
\[
\rho_{E,4}(G_\mbq) \subseteq \GL_2(\mbz/4\mbz)_{\ve = \chi_4} \; \Longleftrightarrow \mbq\left( \sqrt{\gD_E} \right) = \mbq(i).
\]
There is a rational parameter $t$ on the genus zero modular curve $X_{\GL_2(\mbz/4\mbz)_{\chi_4 = \ve}}$ with the property that the forgetful map $X_{\GL_2(\mbz/4\mbz)_{\chi_4 = \ve}} \rightarrow X(1)$ takes the form $t \mapsto j_4(t)$, where
\[
j_4(t) := -t^2 + 1728.
\]
As detailed in \cite{sutherlandzywina}, for any elliptic curve $E$ over $\mbq$ with $j$-invariant $j_E$, we have
\begin{equation} \label{level4containmentintermsofjinvariant}
\rho_{E,4}(G_\mbq) \subseteq \GL_2(\mbz/4\mbz)_{\chi_4 = \ve} \; \Longleftrightarrow \; \exists t_0 \in \mbq \; \text{ with } \; j_E = j_4(t_0).
\end{equation}
In particular, defining the elliptic curve $\mc{E}_4$ over $\mbq(t,D)$ by
\begin{equation} \label{level4genericellipticcurve}
\mc{E}_4 : \; y^2 = x^3 + \frac{108D^2 j_{4}(t)}{1728 - j_{4}(t)} x + \frac{432D^3 j_{4}(t)}{1728 - j_{4}(t)},
\end{equation}
we have that $\rho_{\mc{E}_4,4}(G_{\mbq(t,D)}) \doteq \GL_2(\mbz/4\mbz)_{\chi_4 = \ve}$.  The following lemma summarizes the situation and will be useful in what follows.
\begin{lemma} \label{identifyingthesubfieldslevel4lemma}
For any elliptic curve $E$ over $\mbq$, we have
\begin{equation} \label{arisesasaspecializationlevel4}
\rho_{E,4}(G_\mbq) \, \dot\subseteq \, \GL_2(\mbz/4\mbz)_{\chi_4 = \ve} \; \Longleftrightarrow \; \exists t_0, D_0 \in \mbq \; \text{ with } \; E \simeq_\mbq \mc{E}_4(t_0,D_0),
\end{equation}
where $\mc{E}_4$ is the elliptic curve over $\mbq(t,D)$ defined by \eqref{level4genericellipticcurve}.  Furthermore, when this is the case, we have
\begin{equation} \label{explicitbiquadraticfourthrootofdiscriminant}
\mbq(\mu_4, \gD_E^{1/4}) = \mbq\left( \mu_4,\gD_{\mc{E}_4(t_0,D_0)}^{1/4} \right) = \mbq\left( i, \sqrt{D_0t_0(t_0^2-1728)} \right).
\end{equation}
In particular, when \eqref{arisesasaspecializationlevel4} holds, the subfield $\mbq\left( \mu_4,\gD_E^{1/4} \right) \subseteq \mbq(E[4])$ is either biquadratic or quadratic over $\mbq$.
\end{lemma}
\begin{proof}
The assertion \eqref{arisesasaspecializationlevel4} follows immediately from \eqref{level4containmentintermsofjinvariant}.  The equality \eqref{explicitbiquadraticfourthrootofdiscriminant} follows from
\[
\gD_{\mc{E}_4} = - \left( \frac{2^9 3^6 D^3 (t^2 - 1728) }{t^3} \right)^2,
\]
using the fact that $(-1)^{1/4} = \zeta_8 = \frac{\sqrt{2}}{2} + \frac{\sqrt{2}}{2}i$, and from \eqref{explicitbiquadraticfourthrootofdiscriminant} one sees that $\mbq\left( \mu_4, \gD_E^{1/4} \right)$ is either biquadratic or quadratic over $\mbq$.
\end{proof}

\subsubsection{The level \texorpdfstring{$m = 5$}}

The subgroups
\begin{equation} \label{defofG51andG52}
G_{5,1} := \left\{ \begin{pmatrix} \pm 1 & * \\ 0 & * \end{pmatrix} \right\}, \; G_{5,2} := \left\{ \begin{pmatrix} * & * \\ 0 & \pm 1 \end{pmatrix} \right\} \subseteq \left\{ \begin{pmatrix} * & * \\ 0 & * \end{pmatrix} \right\} \subseteq \GL_2(\mbz/5\mbz)
\end{equation}
correspond to two genus zero modular curves $X_{G_{5,1}}$ and $X_{G_{5,2}}$, each of which is a $2$-fold cover of $X_0(5)$.
As discussed in \cite{zywina}, the maps $\mbp^1(t) \rightarrow \mbp^1(j)$ corresponding to the forgetful maps $X_{G_{5,i}} \longrightarrow X(1)$ are given respectively by the rational functions
\[
j_{5,1}(t) := \frac{(t^4 - 12t^3 + 14t^2 + 12t + 1)^3}{t^5(t^2 - 11t - 1)}, \quad\quad
j_{5,2}(t) :=  \frac{(t^4 + 228t^3 + 494t^2 - 228t + 1)^3}{t(t^2 - 11t - 1)^5}.
\]
We define the elliptic curves $\mc{E}_{5,i}$ over $\mbq(t,D)$ by
\begin{equation} \label{level5genericellipticcurve}
\mc{E}_{5,i} : \; y^2 = x^3 + \frac{108D^2 j_{5,i}(t)}{1728 - j_{5,i}(t)} x + \frac{432D^3 j_{5,i}(t)}{1728 - j_{5,i}(t)} \quad\quad\quad \left( i \in \{ 1, 2 \} \right);
\end{equation}
we have
\begin{equation} \label{genericGaloisimageiscontainedinborelmod5}
\rho_{\mc{E}_{5,i},5}(G_{\mbq(t,D)}) \, \doteq \, G_{5,i} \subseteq \GL_2(\mbz/5\mbz) \quad\quad\quad \left( i \in \{ 1, 2 \} \right).
\end{equation}
The next lemma explicitly characterizes the subfields cut out by the characters
\begin{equation*} \label{defofpsisforlevel5}
\begin{split}
\psi_{5,1}, \; \psi_{5,2}^2\psi_{5,1} : \left\{ \begin{pmatrix} \pm 1 & * \\ 0 & * \end{pmatrix} \right\} &\longrightarrow \{ \pm 1 \} \subseteq \left( \mbz/5\mbz \right)^\times, \\
\psi_{5,2}, \; \psi_{5,1}^2\psi_{5,2} : \left\{ \begin{pmatrix} * & * \\ 0 & \pm 1 \end{pmatrix} \right\} &\longrightarrow \{ \pm 1 \} \subseteq \left( \mbz/5\mbz \right)^\times.
\end{split}
\end{equation*}
\begin{lemma} \label{subfieldsoflevel5lemma}
For each $i \in \{1, 2\}$, let $\mc{E}_{5,i}$ be the elliptic curve over $\mbq(t,D)$ defined by \eqref{level5genericellipticcurve} and let  
\[
\chi_{5,1}^{(i)}, \; \chi_{5,2}^{(i)} : \gal\left( \mbq(t,D)(\mc{E}_{5,i}[5])/\mbq(t,D) \right) \longrightarrow \{ \pm 1 \} \subseteq (\mbz/5\mbz)^\times
\]
denote the restrictions under \eqref{genericGaloisimageiscontainedinborelmod5} of the characters $\chi_{5,1}^{(i)} := \psi_{5,i}$ and $\chi_{5,2}^{(i)} := \psi_{5,i}\psi_{5,3-i}^2$, where $\psi_{5,i}$ are as in \eqref{generaldefinitionofpsisubp}.  We then have
\begin{equation} \label{subfieldsoflevel5lemmaeqn}
\begin{split}
\mbq(t,D)(\mc{E}_{5,1}[5])^{\ker \chi_{5,1}^{(1)}} &= \mbq(t,D)\left( \sqrt{-\frac{2D(t^4 - 12t^3 + 14t^2 + 12t + 1)}{((t^2 + 1)(t^4 - 18t^3 + 74t^2 + 18t + 1)}} \right), \\
\mbq(t,D)(\mc{E}_{5,1}[5])^{\ker \chi_{5,2}^{(1)}} &= \mbq(t,D)\left( \sqrt{-\frac{10D(t^4 - 12t^3 + 14t^2 + 12t + 1)}{((t^2 + 1)(t^4 - 18t^3 + 74t^2 + 18t + 1)}} \right), \\
\mbq(t,D)(\mc{E}_{5,2}[5])^{\ker \chi_{5,1}^{(2)}} &= \mbq(t,D)\left( \sqrt{-\frac{2D(t^4 + 228t^3 + 494t^2 - 228t + 1)}{(t^2+1)(t^4 - 522t - 10006t^2 + 522t + 1)}} \right), \\
\mbq(t,D)(\mc{E}_{5,2}[5])^{\ker \chi_{5,2}^{(2)}} &= \mbq(t,D)\left( \sqrt{-\frac{10D(t^4 + 228t^3 + 494t^2 - 228t + 1)}{(t^2+1)(t^4 - 522t - 10006t^2 + 522t + 1)}} \right).
\end{split}
\end{equation}
\end{lemma}
\begin{proof}
For any elliptic curve $E$ over $\mbq$ with $\rho_{E,5}(G_\mbq) \subseteq B(5)$, define $\langle P \rangle \subseteq E[5]$ and $\langle P' \rangle \subseteq E_{\langle P \rangle}'[5]$ as in \eqref{interpretationofwhenGaloismapsintoBorel} and \eqref{existenceofPprime}.  By \eqref{psisubellsactingonPandPprime} and \eqref{defofG51andG52}, we have
\begin{equation} \label{conditionforG51andG52}
\begin{split}
\rho_{E,5}(G_\mbq) \, \dot\subseteq \, G_{5,1} \; &\Longleftrightarrow \; \exists \text{ a $G_\mbq$-stable } \langle P \rangle \subseteq E[5] \quad\quad\;\;\; \text{ with } \quad [ \mbq( P ) : \mbq ] \leq 2, \\
\rho_{E,5}(G_\mbq) \, \dot\subseteq \, G_{5,2} \; &\Longleftrightarrow \; \begin{matrix} \exists \text{ a $G_\mbq$-stable } \langle P \rangle \subseteq E[5] \text{ and} \\ \exists \text{ a $G_\mbq$-stable  } \langle P' \rangle \subseteq E_{\langle P \rangle}' [5] \end{matrix} \quad \text{ with } \quad [ \mbq( P' ) : \mbq ] \leq 2,
\end{split}
\end{equation}
and the same statement holds when the base field $\mbq$ is replaced by $\mbq(t,D)$.  Furthermore, we have
\begin{equation} \label{kernelsintermsofPlevel5}
\mbq(t,D)\left( \mc{E}_{5,1}[5] \right)^{\ker \chi_{5,1}^{(1)}} = \mbq(t,D)\left( P \right), \quad\quad \mbq(t,D)\left( \mc{E}_{5,2}[5] \right)^{\ker \chi_{5,1}^{(2)}} = \mbq(t,D)\left( P' \right), 
\end{equation}
where $P \in \mc{E}_{5,1}[5]$ and $P' \in (\mc{E}_{5,2})_{\langle P \rangle}'[5]$ are as in \eqref{conditionforG51andG52}.

Using the linear factor of the $5$th division polynomial of $\mc{E}_{5,1}$, we find the point $P_1 = (x_1,y_1) \in \mc{E}_{5,1}[5]$, where
\[
\begin{split}
x_1 &:= -\frac{6D(t^2 - 6t + 1)(t^4 - 12t^3 + 14t^2 + 12t + 1)}{(t^2 + 1)(t^4 - 18t^3 + 74t^2 + 18t + 1)}, \\
y_1 &:=  \frac{216dt(t^4 - 12t^3 + 14t^2 + 12t + 1)}{(t^2 + 1)(t^4 - 18t^3 + 74t^2 + 18t + 1)} \sqrt{\frac{-2D(t^4 - 12t^3 + 14t^2 + 12t + 1)}{(t^2 + 1)(t^4 - 18t^3 + 74t^2 + 18t + 1)}}.
\end{split}
\]
By \eqref{kernelsintermsofPlevel5}, this proves the first equality in \eqref{subfieldsoflevel5lemmaeqn}. We now consider the character $\displaystyle \psi_{5,2}^2 : G_{5,1} \rightarrow \{ \pm 1 \}$, whose value $\psi_{5,2}(g_1)^2$ agrees with $\displaystyle \left( \frac{5}{\det g_1} \right)$, and thus has corresponding fixed field $\mbq(t,D)\left( \sqrt{5} \right)$.  Since $\chi_{5,2}^{(1)} = \psi_{5,2}^2 \psi_{5,1}$, this observation establishes the second equality in \eqref{subfieldsoflevel5lemmaeqn}.  

For the second pair of equalities in \eqref{subfieldsoflevel5lemmaeqn}, we reason as follows.  The $5$th division polynomial associated to $\mc{E}_{5,2}$ has a quadratic factor that is irreducible over $\mbq(t,D)$, and this leads to a $G_{\mbq(t,D)}$-stable cyclic subgroup $\langle P \rangle \subseteq \mc{E}_{5,2}[5]$.  We find that the isogenous elliptic curve $\left( \mc{E}_{5,2} \right)_{\langle P \rangle}' = \mc{E}_{5,2} / \langle P \rangle$ is given by
\[
\begin{split}
\left( \mc{E}_{5,2} \right)_{\langle P \rangle}' : \; y^2 = x^3 &- \frac{67500D^2(t^4 - 12t^3 + 14t^2 + 12t + 1)(t^4 + 228t^3 + 494t^2 - 228t + 1)^2}{(t^2+1)^2(t^4 - 522t^3 - 10006t^2 + 522t + 1)^2}x \\ 
&- \frac{6750000D^3(t^4 - 18t^3 + 74t^2 + 18t + 1)(t^4 + 228t^3 + 494t^2 - 228t + 1)^3}{(t^2+1)^2(t^4 - 522t^3 - 10006t^2 + 522t + 1)^3}.
\end{split}
\]
The $5$th division polynomial of $\left( \mc{E}_{5,2} \right)_{\langle P \rangle}'$ is seen to have a linear factor, which leads to the point $P_2' = (x_2', y_2') \in \left( \mc{E}_{5,2} \right)_{\langle P \rangle}'[5]$, where
\[
\begin{split}
x_2' &:= -150 \frac{D (t^2 - 6t + 1) (t^4 + 228t^3 + 494t^2 - 228t + 1)}{(t^2 + 1) (t^4 - 522t^3 - 10006t^2 + 522t + 1)}, \\
y_2' &:= 27000 \frac{D t (t^4 + 228t^3 + 494t^2 - 228t + 1)}{(t^2 + 1) (t^4 - 522t^3 - 10006t^2 + 522t + 1)} \sqrt{\frac{-2 D (t^4 + 228t^3 + 494t^2 - 228t + 1)}{(t^2 + 1) (t^4 - 522t^3 - 10006t^2 + 522t + 1)}}.
\end{split}
\]
As before, this, together with the fact that for $g_2 \in G_{5,2}$, the value $\psi_{5,1}(g_2)^2$ agrees with $\displaystyle \left( \frac{5}{\det g_2} \right)$ and that $\chi_{5,2}^{(2)} = \psi_{5,1}^2 \psi_{5,2}$, establishes the second two equalities in \eqref{subfieldsoflevel5lemmaeqn}, proving the lemma.
\end{proof}

\subsubsection{The level \texorpdfstring{$m=7$}}

We begin by describing an explicit Weierstrass model $\mc{E}_7$ over $\mbq(t,D)$ that is generic in the sense that its specializations $\mc{E}_7(t_0,D_0)$ give rise to all elliptic curves $E$ over $\mbq$ for which $\rho_{E,7}(G_\mbq) \, \dot\subseteq \, B(7)$, where we recall that
\[
B(7) = \left\{ \begin{pmatrix} * & * \\ 0 & * \end{pmatrix} \right\} \subseteq \GL_2(\mbz/7\mbz)
\]
denotes the Borel subgroup.  We then describe explicitly certain subfields of $\mbq(\mc{E}_7[7])$ that will be useful in the next section.  

Define $j_7(t) \in \mbq(t)$ by
\begin{equation} \label{defofjsub7}
j_7(t) := \frac{(t^2 + 245t + 2401)^3(t^2 + 13t + 49)}{t^7},
\end{equation}
and the elliptic curve $\mc{E}_7$ over $\mbq(t,D)$ by
\begin{equation} \label{level7genericellipticcurve}
\begin{split}
&\mc{E}_7 : \; y^2 = x^3 + D^2 a_{4;7}(t) x + D^3 a_{6;7}(t), \\
& a_{4;7}(t) := \frac{108 j_{7}(t)}{1728 - j_{7}(t)}, \quad\quad a_{6;7}(t) := \frac{432 j_{7}(t)}{1728 - j_{7}(t)}.
\end{split}
\end{equation}
As proved in \cite{zywina}, we have
\begin{equation} \label{genericGaloisimageiscontainedinborelmod7}
\rho_{\mc{E}_{7},7}(G_{\mbq(t,D)}) \, \doteq \, B(7) \subseteq \GL_2(\mbz/7\mbz).
\end{equation}
The next lemma explicitly characterizes the subfields cut out by the quadratic characters $\psi_{7,1}^3$ and $\psi_{7,2}^3$.
\begin{lemma} \label{quadraticsubfieldsoflevel7lemma}
Let $\mc{E}_{7}$ be the elliptic curve over $\mbq(t,D)$ defined by \eqref{level7genericellipticcurve} and let  
\[
\psi_{7,1}^3, \; \psi_{7,2}^3 : \gal\left( \mbq(t,D)(\mc{E}_{7}[7])/\mbq(t,D) \right) \rightarrow \{ \pm 1 \}
\]
denote the restrictions under \eqref{genericGaloisimageiscontainedinborelmod7} of the cubes of the characters defined in \eqref{generaldefinitionofpsisubp}.  We then have
\begin{equation} \label{quadsubfieldsoflevel7lemmaeqn}
\begin{split}
\mbq(t,D)(\mc{E}_{7}[7])^{\ker \psi_{7,1}^3} &= \mbq(t,D)\left( \sqrt{\frac{14D(t^2 + 13t + 49)(t^2 + 245t + 2401)}{(t^4 - 490t^3 - 21609t^2 - 235298t - 823543)}} \right), \\
\mbq(t,D)(\mc{E}_{7}[7])^{\ker \psi_{7,2}^3} &= \mbq(t,D)\left( \sqrt{- \frac{2D(t^2 + 13t + 49)(t^2 + 245t + 2401)}{(t^4 - 490t^3 - 21609t^2 - 235298t - 823543)}} \right).
\end{split}
\end{equation}
\end{lemma}
\begin{proof}
A computation reveals that the $7$th division polynomial of $\mc{E}_7$ has the cubic factor
\begin{equation} \label{cubicfactorofpsisub7}
\begin{split}
x^3 &- \frac{126D(t^2 + 13t + 49)(t^2 + 245t + 2401)}{(t^4 - 490t^3 - 21609t^2 - 235298t - 823543)}x^2 \\ &+ \frac{108D^2(t^2 + 13t + 49)(t^2 + 245t + 2401)^2(33t^2 + 637t + 2401)}{(t^4 - 490t^3 - 21609t^2 - 235298t - 823543)^2}x \\
&- \frac{216D^3(t^2 + 13t + 49)(t^2 + 245t + 2401)^3(881t^4 + 38122t^3 + 525819t^2 + 3058874t + 5764801)}{7(t^4 - 490t^3 - 21609t^2 - 235298t - 823543)^3},
\end{split}
\end{equation}
which is irreducible over $\mbq(t,D)$ and whose discriminant is in $\left( \mbq(t,D)^\times \right)^2$.  Writing this polynomial in the form $f(x) = (x - \ga_1)(x - \ga_2)(x - \ga_3)$, we have that $\mbq(t,D)\left( \ga_1 \right)$ is cyclic cubic over $\mbq(t,D)$ and that the point 
\begin{equation} \label{defofPforlevel7}
P := \left( \ga_1, \sqrt{ \ga_1^3 + D^2 a_{4;7}(t) \ga_1 + D^3 a_{6;7}(t) } \right) 
\end{equation}
generates a cyclic submodule $\langle P \rangle \subseteq \mc{E}_7[7]$ on which $G_{\mbq(t,D)}$ acts through the eigenfunction $\psi_{7,1}$ via 
\begin{equation} \label{Pfixedbykernelofpsisub7supone}
\gs : P \mapsto \left[ \psi_{7,1}(\gs) \right] P.
\end{equation}
We have
\begin{equation} \label{whatmbqtdPlookslike}
\begin{split}
\mbq(t,D)\left( P \right) &= \mbq(t,D)\left( \ga_1, \sqrt{ \ga_1^3 + D^2 a_{4;7}(t) \ga_1 + D^3 a_{6;7}(t) } \right) \\
&\supseteq \mbq(t,D)\left( \ga_1, \sqrt{ \prod_{j = 1}^3 (\ga_j^3 + D^2 a_{4;7}(t) \ga_j + D^3 a_{6;7}(t)) } \right);
\end{split}
\end{equation}
a computation using the third equation in \eqref{symmetricpolynomialequalities} shows that 
\[
\begin{split}
\mbq(t,D)\left(  \sqrt{ \prod_{j=1}^3 (\ga_j^3 +& D^2 a_{4;7}(t) \ga_j + D^3 a_{6;7}(t) ) } \right) =\\
&\mbq(t,D)\left( \sqrt{\frac{14D(t^2 + 13t + 49)(t^2 + 245t + 2401)}{(t^4 - 490t^3 - 21609t^2 - 235298t - 823543)}} \right),
\end{split}
\]
establishing the first equality in \eqref{quadsubfieldsoflevel7lemmaeqn} (and also showing that we have equality in \eqref{whatmbqtdPlookslike}).  The second equality follows from the fact that the product $\psi_{7,1}^3(g) \psi_{7,2}^3(g)$ agrees with $\displaystyle \left( \frac{-7}{\det g} \right)$, whose fixed field is $\mbq(t,D)\left( \sqrt{-7} \right)$.
\end{proof}

Our next lemma explicitly characterizes the subfields cut out by the cubic characters $\psi_{7,1}^2, \psi_{7,2}^2$ and $\psi_{7,1}^2\psi_{7,2}^4$, where $\psi_{7,i}$ is defined as in \eqref{generaldefinitionofpsisubp}.

We define the polynomials
\begin{equation} \label{defoffsubTRandRprime}
\begin{split}
f_{\cyc,7}^+(X) &:= X^3 + X^2 - 2X - 1, \\
f_T(X) &:= X^3 - T_1(t) X^2 + T_2(t) X - T_3(t), \\
f_R(X) &:= X^3 - R_1(t) X^2 + R_2(t) X - R_3(t), \\
f_{R'}(X) &:= X^3 - R_1(t) X^2 + R_2(t) X - R_3'(t),
\end{split}
\end{equation}
where
\begin{equation} \label{defoffsubXcoefficients}
\begin{split}
T_1(t) &:= -21 ( t^2 + 13t + 49 ), \\
T_2(t) &:= 3 (t^2 + 13t + 49) (33 t^2 + 637t + 2401), \\
T_3(t) &:= -\frac{1}{7} (t^2 + 13t + 49) ( 881 t^4 + 38122 t^3 + 525819 t^2 + 3058874 t + 5764801 ), \\
R_1(t) &:= 21 (t^2 + 13t + 49), \\
R_2(t) &:= -21(t^2 + 13t + 49) (9t^2 - 91 t - 343), \\
R_3(t) &:= -7 (t^2 + 13t + 49) ( 223t^4 + 3542t^3 + 3381t^2 - 62426 t - 117649 ), \\
R_3'(t) &:= - (t^2 + 13t + 49) (3289 t^4 + 24794 t^3 + 23667 t^2 - 436982 t - 823543 ).
\end{split}
\end{equation}
\begin{lemma}
Let $\mc{E}_7$ be be the elliptic curve over $\mbq(t,D)$ defined by \eqref{level7genericellipticcurve} and let  
\[
\psi_{7,1}^2, \psi_{7,2}^2 : \gal\left( \mbq(t,D)(\mc{E}_{7}[7])/\mbq(t,D) \right) \longrightarrow \left( (\mbz/7\mbz)^\times \right)^2 \simeq \mu_3
\]
denote the restrictions under \eqref{genericGaloisimageiscontainedinborelmod7} of the squares of the characters $\psi_{7,i}$ defined in \eqref{generaldefinitionofpsisubp}.  Let us denote by $\mbq(t,D)_{f_{\cyc,7}^+}$, $\mbq(t,D)_{f_T}$, $\mbq(t,D)_{f_R}$ and $\mbq(t,D)_{f_{R'}}$ the splitting fields over $\mbq(t,D)$ of the polynomials $f_{\cyc,7}^+(X)$, $f_T(X)$, $f_R(X)$ and $f_{R'}(X)$, respectively, where these polynomials are defined by \eqref{defoffsubTRandRprime} and \eqref{defoffsubXcoefficients}.
We then have
\begin{equation} \label{cubicsubfieldsoflevel7lemmaeqn}
\begin{split}
&\mbq(t,D)\left( \mc{E}_{7}[7] \right)^{\ker \psi_{7,1}^2} = \mbq(t,D)_{f_T}, \quad
\mbq(t,D)\left( \mc{E}_{7}[7] \right)^{\ker \psi_{7,1}^2 \psi_{7,2}^2} = \mbq(t,D)_{f_{\cyc,7}^+}, \\
&\mbq(t,D)\left( \mc{E}_{7}[7] \right)^{\ker \psi_{7,2}^2} = \mbq(t,D)_{f_R}, \quad \mbq(t,D)(\mc{E}_{7}[7])^{\psi_{7,1}^2\psi_{7,2}^4} = \mbq(t,D)_{f_{R'}}.
\end{split}
\end{equation}
\end{lemma}
\begin{proof}
Since $\rho_{\mc{E}_7,7}\left(G_{\mbq(t,D)}\right) = B(7)$, it is straightforward to see that $\mbq(t,D)\left( \mc{E}_7[7] \right)$ has exactly $4$ cyclic cubic subfields.  Furthermore, we have
\[
\mbq(t,D)\left(\mc{E}_7[7] \right)^{\ker \psi_{7,1}^2 \psi_{7,2}^2} = \mbq(t,D)\left( \mu_7 \right)^+ \subseteq \mbq(t,D)\left( \mu_7 \right),
\]
the first equality above following from the fact that $\psi_{7,1}(g)\psi_{7,2}(g) = \det g$, which implies that the fixed field of $\ker \psi_{7,1}^2 \psi_{7,2}^2$ is the maximal real subfield $\mbq(t,D)\left( \mu_7 \right)^+$, i.e. the unique subfield of $\mbq(t,D)\left( \mu_7 \right)$ that is cyclic cubic over $\mbq(t,D)$.  We have $\mbq(t,D)\left( \mu_7 \right)^+ = \mbq(t,D)\left( \zeta_7 + \zeta_7^{-1} \right)$, so $\mbq(t,D)\left( \mu_7 \right)^+$ is the splitting field of
\[
f_{\cyc,7}^+(X) = X^3 + X^2 - 2X - 1,
\]
the minimal polynomial over $\mbq(t,D)$ of the generator $\zeta_7 + \zeta_7^{-1}$.  This establishes the second equality in \eqref{cubicsubfieldsoflevel7lemmaeqn}.

For the equality in \eqref{cubicsubfieldsoflevel7lemmaeqn} involving the fixed field of $\ker \psi_{7,1}^2$, we reason as follows.  By \eqref{Pfixedbykernelofpsisub7supone} and \eqref{defofPforlevel7}, we see that $\mbq(t,D)\left( \mc{E}_7[7] \right)^{\ker \psi_{7,1}} = \mbq(t,D)(P)$, where $\langle P \rangle \subseteq \mc{E}_7[7]$ is a cyclic $G_{\mbq(t,D)}$-stable subgroup, and since this extension is cyclic of degree $6$ over $\mbq(t,D)$, it follows that
\begin{equation} \label{generatedbyxcoordinateofP}
\mbq(t,D)\left( \mc{E}_7[7] \right)^{\ker \psi_{7,1}^2} = \mbq(t,D)\left( \ga_1 \right).
\end{equation}
where $\ga_1$ is the $x$-coordinate of $P$.
Finally, the substitution $x = -\frac{6D(t^2 + 245t + 2401)}{(t^4 - 490t^3 - 21609t^2 - 235298t - 823543)}X$ transforms the cyclic cubic polynomial \eqref{cubicfactorofpsisub7} into $\left( -\frac{6D(t^2 + 245t + 2401)}{(t^4 - 490t^3 - 21609t^2 - 235298t - 823543)} \right)^3f_T(X)$, and the first equality in \eqref{cubicsubfieldsoflevel7lemmaeqn} follows.

The discriminants $\gD_T$ and $\gD_{\cyc,7}^+$ associated to the polynomials $f_T(X)$ and $f_{\cyc,7}^+(X)$ satisfy
\[
\sqrt{\gD_T} = \frac{2^63^3t^4(t^2 + 13t + 49)}{7}, \quad\quad \sqrt{\gD_{\cyc,7}^+} = 7.
\]
Applying Lemma \ref{gettingattheothercubicfieldslemma}, we find that $\mbq(t,D)_{f_R}$ and $\mbq(t,D)_{f_{R'}}$ are the remaining two cyclic cubic subfields of $\mbq(t,D)\left( \mc{E}_7[7] \right)$.  To see which subfield is which, we first note that, just as in \eqref{generatedbyxcoordinateofP}, we have
\[
\mbq(t,D)\left( \mc{E}_7[7] \right)^{\ker \psi_{7,2}^2} = \mbq(t,D)\left( x(P') \right),
\]
where $P' \in (\mc{E}_7)_{\langle P \rangle}'$ is any generator of a cyclic $G_{\mbq(t,D)}$-stable subgroup $\langle P' \rangle \subseteq (\mc{E}_7)_{\langle P \rangle}'[7]$.  A direct computation reveals that the isogenous curve $\left(\mc{E}_7\right)_{\langle P \rangle}'$ has Weierstrass equation
\[
\begin{split}
\left(\mc{E}_7\right)_{\langle P \rangle}' : \; y^2 = x^3 - &\frac{2^23^37^4D^2(t^2 + 5t + 1)(t^2 + 13t + 49)(t^2 + 245t + 2401)^2}{(t^4 - 490t^3 - 21609t^2 - 235298t - 823543)^3}x \\
- &\frac{2^43^37^6D^3(t^2 + 13t + 49)(t^2 + 245t + 2401)^3(t^4 + 14t^3 + 63t^2 + 70t - 7)}{(t^4 - 490t^3 - 21609t^2 - 235298t - 823543)^3},
\end{split}
\]
and that its $7$th division polynomial has the cubic factor
\[
\begin{split}
x^3 + &\frac{2\cdot 3^27^2D(t^2 + 13t + 49)(t^2 + 245t + 2401)}{(t^4 - 490t^3 - 21609t^2 - 235298t - 823543)}x^2 \\
+ &\frac{2^23^37^4D^2(t^2 + 13t + 33)(t^2 + 13t + 49)(t^2 + 245t + 2401)^2}{(t^4 - 490t^3 - 21609t^2 - 235298t - 823543)^2}x \\
+ &\frac{2^33^37^6D^3(t^2 + 13t + 49)(t^2 + 245t + 2401)^3(t^4 + 26t^3 + 219t^2 + 778t + 881)}{(t^4 - 490t^3 - 21609t^2 - 235298t - 823543)^3}.
\end{split}
\]
Finally, applying Lemma \ref{settingcubicfieldsequaltoeachotherlemma} (and some extensive, tedious calculations), we see that the splitting field of this polynomial agrees with the splitting field of $f_R(X)$, and this finishes the proof.
\end{proof}